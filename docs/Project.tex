%
% File acl2018.tex
%
%% Based on the style files for ACL-2017, with some changes, which were, in turn,
%% Based on the style files for ACL-2015, with some improvements
%%  taken from the NAACL-2016 style
%% Based on the style files for ACL-2014, which were, in turn,
%% based on ACL-2013, ACL-2012, ACL-2011, ACL-2010, ACL-IJCNLP-2009,
%% EACL-2009, IJCNLP-2008...
%% Based on the style files for EACL 2006 by 
%%e.agirre@ehu.es or Sergi.Balari@uab.es
%% and that of ACL 08 by Joakim Nivre and Noah Smith

\documentclass[11pt,a4paper]{article}
\usepackage[hyperref]{acl2018}
\usepackage{times}
\usepackage{latexsym}

\usepackage{url}

%\aclfinalcopy % Uncomment this line for the final submission
%\def\aclpaperid{***} %  Enter the acl Paper ID here

%\setlength\titlebox{5cm}
% You can expand the titlebox if you need extra space
% to show all the authors. Please do not make the titlebox
% smaller than 5cm (the original size); we will check this
% in the camera-ready version and ask you to change it back.

\newcommand\BibTeX{B{\sc ib}\TeX}

\title{Pun and Multiple Meaning Detection}

\author{Chaim Schendowich \\
  300307659 \\
  {\tt chysch@gmail.com} \\\And
  Anna Sokolov \\
  306616178 \\
  {\tt nannasokolov@gmail.com} \\}

\date{}

\begin{document}
\maketitle
\begin{abstract}
Creative speakers tend occasionally to instill in their words puns and multiple meanings. When the multiple meaning arises from the sound of the words, namely is caused by saying sounds that can be interpreted as two or more different combinations of words which all make sense in the given context, it is called a homophonic pun. The intention of this project is detection of such puns in given sentences if they exist and indication if not.
\end{abstract}

\section{Introduction}

\section{Method}

\section{Results}

\section{Related Works}

Although much work has been done on classification and detection of humor in general and puns in particular, we did not find written material on detection of homophonic puns.

\citet{yang_lavie_dyer_hovy_2015} searched for humor anchors in sentences and detected humor through them. They did not search for puns specifically rather semantically humorous constructs. Even though their emphasis was not puns it is important to note that the idea of working through anchors is definately relevant in this context - we used a similar idea in the Threshold hyper-parameter.

Various groups worked on pun detection and understanding. \citet{miller_gurevych_2015} used Word Sense Disambiguation (WSD) to find the meanings of puns in corpora of homographic puns with only two meanings and exactly one pun word per sentence. Later \citet{miller_hempelmann_gurevych_2017} also continued the research comparing various types of WSD to see which method has better success not only disambiguating the puns but also detecting their existance and locating them. Unfortunately, the restrictions they put on their research meant that they were not dealing with homophonic puns.

The closest study we found was done by \citet{jaech_koncel-kedziorski_ostendorf_2016}. They cited various people who classified types of puns and analyzed puns theoretically. Their research was into understanding homophonic puns in sentences where the locations of the puns were already given. Since we are more interested in detecting if a homophonic pun exists and where, that approach was also not enough.

\section{Conclusions and Future Work}

% include your own bib file like this:
%\bibliographystyle{acl}
%\bibliography{acl2018}
\bibliography{ProjectBiblio}
\bibliographystyle{acl_natbib}

\end{document}
